

\chapter{Introduction}



\section{Panoramic Image Storage}


\subsection{Equirectangular Image}


\subsection{Cube Map}


\subsection{Geodesic Grids}



\subsection{Tangent Image}


Spherical Coordinate system.



\subsubsection{Gnomonic Projection}

\textbf{Sphere 2 Perspective Image}

Set the point $S$ as the centre point of the projection, who longitude and latitude is $(\theta, \phi)$

For a projection with the central point $(\theta_0, \phi_0)$,



\begin{itemize}
	\item Ohttps://mathworld.wolfram.com/GnomonicProjection.html
	\item Another entry in the list
\end{itemize}


\section{Optical Flow}

\begin{center}
	\begin{tabular}{||c c ||} 
		\hline
		Notice & Meaning  \\ [0.5ex] 
		\hline\hline
		$\textbf{f}$ & optical flow  \\ 
		\hline
		$(u,v)$ & the position of pixel   \\
		\hline
		$f(u,v)$ & the pixel's optical flow of image $I_{src}$ \\
		\hline
		$\textbf{I}$ & the RGB image  \\ [1ex] 
		\hline
	\end{tabular}
\end{center}

The optical flow $\textbf{f}$ is estimated pixel displacement vector $f(u,v)$ for each pixel $(u,v)$ in a sequence image pair from source image $I_{src}$ to target image $I_{tar}$.
The $f(u,v)$ compose with $f_1(u,v)$ and $f_2(u,v)$, which is the displacement in image's horizontal direction $x$ and vertical direction $y$ respectively.
Optical flow 

$I_{tar}(u',v')$ is corresponding the pixel $I_{src}(u + f_1(u,v), v + f_2(u,v))$.


\subsection{360 Optical Flow Expression}

The panoramic image's $x$ and $y$ is continual, in this way it's reasonable the pixel coordinate 0 exceed the range of image pixel $([0, image\_width), [0, image\_height))$. 

The range of optical flow is 

So the 360 optical flow is different with perspective image optical flow in the 

, 360 optical flow is  in 

So there are two-kinds of expression.

One kind optical flow is just in target image range $[0, image_width - 1]$, which do not take the wrap around into account.

So the range of optical flow is 
Another's target image range is $[- n * image_width, + n * image_width]$

Warp around


\section{Interpolation}


