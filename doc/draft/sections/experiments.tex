\chapter{Experiments}


\section{Dataset}

It's very hard to measure the real word optical flow information, especially the equirectangular image. 
And currently do not have any public optical flow equirectangular image dataset.
So we evaluate our method estimated optical flow quantity in the synthetic equirectangular optical flow dataset.
Meanwhile to analysis the performance we use SSIM e.t.c metrics to evaluate the optical flow warped the equirectangular image sequence in both real world dataset and synthetic dataset.


\subsection{Synthetic Dataset}

The rasterization render method OpenGL used to synthesize the ground truth optical flow.
The equirectangular optical flow generation algorithm introduction in Section. \ref{sec:app:panoof}.
Few public indoor 3D datasets are hired to render the ground truth data.

Although, the 3D computer graphics software toolsets, e.g. Blender have functions to estimate the object motion information, e.g motion vector of Blender. 
The rendered object motion used for motion blur e.t.c VFX is not accurate enough. 
There are obviously artefacts at the boundary of moving objects, Figure. ~\ref{fig:exp:blendermv}. 

\begin{figure}[hbt!]
	\centering
	\includegraphics[width=\linewidth]{example-image-a}
	\caption{Blender Motion Vector}
	\label{fig:exp:blendermv}
\end{figure}

\textbf{Replica 360} ~\citet{replica19arxiv}


\textbf{Matterport3D} ~\citet{Matterport3D}


\subsection{Real World Dataset}

We test our method on the real world dataset with warped image sequence.

\section{Comparison}

We compare our method with following state of the art methods, FlowNet2~\cite{ilg2017flownet}, Dense Inverse Search (DIS) optical flow~\cite{kroeger2016fast} and PWC-Net~\cite{sun2018pwc}. 
The FlowNet2~\cite{ilg2017flownet} implementation is NVIDIA official release code \href{https://github.com/NVIDIA/flownet2-pytorch}{flownet2-pytorch}.
The PWC-Net~\cite{sun2018pwc}'s code is \href{https://github.com/NVlabs/PWC-Net}{NVlabs/PWC-Net}

\subsection{Quantity Analysis}

The metrics AAE, EPE and RMS is used for the quantity analysis.

The average endpoint error (\textbf{EPE}):  ~\cite{??}
EPE is the absolute error of all the pixels. 

\begin{equation}\label{equ_exp_epe}
E_{EPE} = \frac{1}{n} \sum_{i \in \Omega}\sqrt{(u_e^i - u_g^i)^2 + (v_e^i - v_g^i)^2}
\end{equation}

The average angular error (\textbf{AAE}):~\cite{??}
AAE is a metric to measure the deviation of angle:

\begin{equation}\label{equ_exp_aae}
	E_{AAE} = \frac{1}{n} \sum_{i \in \Omega}arctan(\frac{u^i_e v^i_g - u^i_g v^i_e}{u^i_e u^i_g + v^i_e v^i_g})
\end{equation}

The root mean square error (\textbf{RMSE}):~\cite{??}

\begin{equation}\label{equ_exp_rmse}
	E_{RMSE} = \sqrt{\frac{1}{n} \sum_{i \in \Omega}((u_e^i - u_g^i)^2 + (v_e^i - v_g^i)^2)}
\end{equation}



%\begin{tikzpicture}
%	% Read the file
%	% if needed, define column separators by pgfplotstableread[col sep=...]{}{}
%	\pgfplotstableread{data.csv}{\Data}
%	\begin{groupplot}[
%			group style = {group size = 2 by 1},
%			height = 10cm,
%			width = 10cm,
%			ymin = 500,
%			xticklabels from table = {\Data}{Day}, xtick = {1,...,19},
%			xticklabel style = {rotate = 90, xshift = -0.8ex, anchor = mid east}
%		]
%		\nextgroupplot[xmin = 1, xmax = 19, ymax = 900]
%		\addplot[very thick] table [x = Number, y = Orders] {\Data};
%	\end{groupplot}
%\end{tikzpicture}


\begin{table}[h!]
	\centering
	\begin{tabular}{ c | c | c | c | c | c | c }
		\hline
		     & \multicolumn{3}{c|}{FlowNet2} & \multicolumn{3}{c}{PWCNet} \\
		\hline
			 & AAE & EPE & RMS & AAE & EPE & RMS \\
		\hline
		apartment\_0 & - & - & -  & - & - & -  \\ 
		\hline
		hotel\_0 & - & - & -  & - & - & -  \\ 
		\hline
		office\_0 & - & - & - & - & - & -  \\ 
		\hline
		office\_4 & - & - & -  & - & - & -  \\ 
		\hline
		room\_0 & - & - & - & - & - & -  \\ 
		\hline
		room\_1 & - & - & -  & - & - & -  \\ 
		\hline\hline
			& \multicolumn{3}{c|}{DIS} & \multicolumn{3}{c}{Ours} \\
		\hline
			& AAE & EPE & RMS & AAE & EPE & RMS \\
		\hline
		apartment\_0 & - & - & -  & - & - & -  \\ 
		\hline
		hotel\_0 & - & - & -  & - & - & -  \\ 
		\hline
		office\_0 & - & - & - & - & - & -  \\ 
		\hline
		office\_4 & - & - & -  & - & - & -  \\ 
		\hline
		room\_0 & - & - & - & - & - & -  \\ 
		\hline
		room\_1 & - & - & -  & - & - & -  \\ 
		\hline
	\end{tabular}
	\caption{Replica dataset}
	\label{fig:exp:quality}
\end{table}

\subsection{Quality Analysis}





