

\chapter{Approach}

\section{Panoramic Optical Flow}


\section{Panoramic Optical Flow Synthetic Dataset}

The real word photo image is hard to estimate accurate optical flow.
So the synthetic method is commonly used to generate the ground truth RGB images sequence and optical flow.


\subsection{Overview}

The traditional OpenGL pipeline don't support the panoramic camera model and optical flow generation.
For synthesising ground truth RGB images and optical flow, we implemented and hired an panoramic camera model.

The camera model composes with 3 parts:
\begin{itemize}
	\item Equirectangular projection;
	\item Warp Around Processing;
	\item Optical flow estimation;
\end{itemize}

\subsection{Equirectangular Projection}

The panoramic camera model render the 3D mesh to a panoramic image with the equirectangular projection.
To achieve the projection, the geometry shader transform the 3D mesh from cartesian coordinate system to spherical coordinate system.
The projection 

Furthermore, we use Replica to demonstrate. 
The Replica dataset coordinate system show as the Fig.~\ref{fig:approach:coord_hotel_00}.

\begin{figure}[hbt!]
	\centering
	\includegraphics[width=\linewidth]{images/synthetic_optical_flow/coord_hotel_00.png}
	\caption{A boat.}
	\label{fig:approach:coord_hotel_00}
\end{figure}

And the coordinate system used in geometry OpenGL shader shown as Fig.~\ref{fig:approach:geometry_cs}.
The forward is $+z$, up is $+y$ and left is $+x$.
Meanwhile, the $\theta$ and $$

\begin{figure}[hbt!]
	\centering
	\includegraphics[width=\linewidth]{images/synthetic_optical_flow/coord_hotel_00.png}
	\caption{A boat.}
	\label{fig:approach:geometry_cs}
\end{figure}


\subsection{Warp Around Processing}




\subsubsection{Cross Image Boundary}


\subsubsection{Poles}


\subsection{Optical Flow Synthesis}



\subsubsection{Occlusion}


