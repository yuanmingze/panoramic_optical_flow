\section{Method}\label{sec:method}

\subsection{Overview}


\begin{figure}[hbt!]
	\centering
	\includegraphics[width=0.4\linewidth]{example-image-a}
	\caption{method pipeline.}
\label{fig:method:pipeline}
\end{figure}


% technicial pipline

\subsection{Optical Flow}

\begin{center}
	\begin{tabular}{ c | c } 
		\hline
		Notice & Meaning  \\ [0.5ex] 
		\hline\hline
		$\textbf{f}$ & optical flow  \\ 
		\hline
		$(u,v)$ & the position of pixel   \\
		\hline
		$f(u,v)$ & the pixel's optical flow of image $I_{src}$ \\
		\hline
		$\textbf{I}$ & the RGB image  \\ [1ex] 
		\hline
	\end{tabular}
\end{center}

The optical flow $\textbf{f}$ is estimated pixel displacement vector $f(u,v)$ for each pixel $(u,v)$ in a sequence image pair from source image $I_{src}$ to target image $I_{tar}$.
The $f(u,v)$ compose with $f_1(u,v)$ and $f_2(u,v)$, which is the displacement in image's horizontal direction $x$ and vertical direction $y$ respectively.
Optical flow 

$I_{tar}(u',v')$ is corresponding the pixel $I_{src}(u + f_1(u,v), v + f_2(u,v))$.


\subsection{360 Optical Flow Expression}

The panoramic image's $x$ and $y$ is continual, in this way it's reasonable the pixel coordinate 0 exceed the range of image pixel $([0, image\_width), [0, image\_height))$. 

The range of optical flow is 

So the 360 optical flow is different with perspective image optical flow in the 

, 360 optical flow is  in 

So there are two-kinds of expression.

One kind optical flow is just in target image range $[0, image_width - 1]$, which do not take the wrap around into account.

So the range of optical flow is 
Another's target image range is $[- n * image_width, + n * image_width]$

Warp around




\subsection{Warp Around}




\subsubsection{Cross Image Boundary}


\subsubsection{Poles}


\subsection{Occlusion}


\subsection{ERP Image}



\subsection{Panoramic Optical Flow}


\textbf{Regular icosahedron}
a convex polyhedron with 20 faces, 30 edges and 12 vertices. 

Reference:
\href{https://en.wikipedia.org/wiki/Regular_icosahedron}{wiki}
\href{https://mathworld.wolfram.com/GnomonicProjection.html}{Gnomonic Projection}
\href{https://mathworld.wolfram.com/RegularIcosahedron.html}{Weisstein, Eric W}
\href{https://math.wikia.org/wiki/Icosahedron}{Weisstein, Eric W}

\href{https://en.wikipedia.org/wiki/Gnomonic_projection}{Weisstein, Eric W}
\href{https://www.imo.net/observations/methods/visual-observation/minor/gnomonic/}{Weisstein, Eric W}

