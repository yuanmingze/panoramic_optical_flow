\section{Approach}\label{sec:approach}

\begin{figure}[hbt!]
	\centering
	\includegraphics[width=0.95\linewidth]{images/method_pipeline_0.pdf}
	\caption{method pipeline. \todolist{Refine.}}
\label{fig:approach:pipeline}
\end{figure}

\todolist{Why use ERP Image representation the panoramic image?}

\todolist{Overview our method, including steps.}

As shown in Fig.~\ref{fig:approach:pipeline}, the method compose with 3 steps:

\begin{enumerate}
	\item Estimate ERP images optial flow to align the target ERP images to srouce image with the estimated rotation;
	\item Project the ERP aligned target ERP image to with CubeMap images, and estimate optical flow for each face, then stich them. And ;
	\item Porject the CubeMap aligned target ERP image with Icosahdron, ;
	\item Accumulate all step optical flow together.
\end{enumerate}

Two Technical pipeline:
\todolist{Rough align the images, with roation.}

\todolist{Rough and warp based method. Why not warp base method? CubeMap warp the target image, and Ico compute the cubemap warpped image. Finally, accumualte all optical flow together.}

The optical flow $\textbf{f}$ is estimated pixel displacement vector $f(u,v)$ for each pixel $(u,v)$ in a sequence image pair from source image $I_{src}$ to target image $I_{tar}$.
The $f(u,v)$ compose with $f_1(u,v)$ and $f_2(u,v)$, which is the displacement in image's horizontal direction $x$ and vertical direction $y$ respectively.
Optical flow 

$I_{tar}(u',v')$ is corresponding the pixel $I_{src}(u + f_1(u,v), v + f_2(u,v))$.

\textbf{360 Optical Flow Expression (Wrap-around)}
\todolist{GT data how to process the Warp Around, Cross Image Boundary.Why use the without warp around optical flow?}

The panoramic image's $x$ and $y$ is continual, in this way it's reasonable the pixel coordinate 0 exceed the range of image pixel $([0, image\_width), [0, image\_height))$. 

The range of optical flow is 

So the 360 optical flow is different with perspective image optical flow in the 

, 360 optical flow is  in 

So there are two-kinds of expression.

One kind optical flow is just in target image range $[0, image_width - 1]$, which do not take the wrap around into account.

So the range of optical flow is 
Another's target image range is $[- n * image_width, + n * image_width]$


\begin{table}
	\begin{center}
		\begin{tabular}{ c | c } 
			\hline
			Notice & Meaning  \\ [0.5ex] 
			\hline\hline
			$\textbf{f}$ & optical flow  \\ 
			\hline
			$(u,v)$ & the position of pixel   \\
			\hline
			$f(u,v)$ & the pixel's optical flow of image $I_{src}$ \\
			\hline
			$\textbf{I}$ & the RGB image  \\ [1ex] 
			\hline
		\end{tabular}
	\end{center}
	\caption{Method pipeline. \todolist{Notations.}}
	\label{tab:approach:notation}
\end{table}

\subsection{Projection \& Stitching}

\textbf{Genometic Projection.}

How the gnomomic projection project the 

Set the tangent point $S$ as the centre point of the projection, who longitude and latitude is $(\theta_0, \phi_0)$,
The sphercial points is $(\theta, \phi)$

%Azimuthal equidistant projection:
%- https://en.wikipedia.org/wiki/Azimuthal_equidistant_projection
%- https://fr.maplesoft.com/applications/view.aspx?SID=3583&view=html
%- https://casa.nrao.edu/aips2_docs/memos/107/node2.html
%- https://casa.nrao.edu/aips2_docs/memos/107/node2.html#SECTION00021100000000000000
%- http://www.geography.hunter.cuny.edu/~jochen/GTECH361/lectures/lecture04/concepts/Map%20coordinate%20systems/Perspective.htm


\textbf{Tangent/Center points selection.}
Use cubemap or Regular icosahedron, which is a convex polyhedron with 20 faces, 30 edges and 12 vertices.

%Reference:
%\href{https://en.wikipedia.org/wiki/Regular_icosahedron}{wiki}
%\href{https://mathworld.wolfram.com/GnomonicProjection.html}{Gnomonic Projection}
%\href{https://mathworld.wolfram.com/RegularIcosahedron.html}{Weisstein, Eric W}
%\href{https://math.wikia.org/wiki/Icosahedron}{Weisstein, Eric W}
%
%\href{https://en.wikipedia.org/wiki/Gnomonic_projection}{Weisstein, Eric W}
%\href{https://www.imo.net/observations/methods/visual-observation/minor/gnomonic/}{Weisstein, Eric W}


\textbf{Panoramic image Projection}

project the panoramic image to perspective images.


\textbf{Perspective Image Stitch}

\todolist{How to compute the weight and blending? Euqation.}
\todolist{Visualized different weight and blend result.}


\subsection{Image \& Optical Flow Alignment}

Inspired by the warp base optical flow \cite{?}, and coarst to fine 

Use panoramic images optical flow estimation rotation to roughly align the two images.

\todolist{How to convert the rough optical flow to rotation?}

The image warp, the rotation to warp the image, alignt the 

The optical flow warping, to warp the final optical flow to restore the original images optical flow.
