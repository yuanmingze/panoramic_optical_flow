\section{Approach}\label{sec:approach}

\begin{figure}[hbt!]
	\centering
	\includegraphics[width=0.95\linewidth]{images/method_pipeline_1.pdf}
	\caption{Method overview. The input is a pair of equirectangular images sequence $I_t$ and $I_{t+1}$, output is the optical flow $\mathcal{F}_t$. The $\oplus$ and  $\ominus$ are the image forward rotation operation and optical flow backward rotation operation ~\cref{sec:approach:warping}, respectively. The green box estimate the rotation from optical flow motion vector.}
	\label{fig:approach:pipeline}
\end{figure}

The method proposes a global rotation warping and tangent image based panoramic optical flow method to mitigate the equirectangular image's distortion and pixel's long span motion, meanwhile, the backbone perspective optical flow method can generalize to any  off-the-shelf optical flow method.

% overview method
As shown in the overview diagram ~\cref{fig:approach:pipeline} our method compose of 3 steps:
1) Estimate strightly the panoramic optical flow $\bar{\mathcal{F}}_t$ from $I_{t}$ to ${I_{t+1}}$, then rotate the $I_{t+1}$ toward to $I_{t}$ with the $\bar{R}$ esitmation form $\mathcal{F}_t$ to generate ${\bar{I}}_{t+1}$~\cref{sec:approach:warping};
2) Estimate the panoramic optical flow ${\hat{\mathcal{F}}}_t$ from $I_{t}$ to ${\bar{I}}_{t+1}$ with cubemap optical flow method ~\cref{sec:approach:projstit}, then rotate the  ${\bar{I}}_{t+1}$ with the $\hat{R}$ esitmated from ${\hat{\mathcal{F}}}_t$ to generate ${\hat{I}}_{t+1}$;
3) Estimate $\tilde{\mathcal{F}}_t$ from $I_{t}$ to ${\hat{I}}_{t+1}$ with icoshedron optical flow mathod~\cref{sec:approach:warping}, then backward rotate warp the $\tilde{\mathcal{F}}_t$ with $\bar{R}$ and $\hat{R}$ to generate the panoramic optical flow $\mathcal{F}_t$.


% 
Comparing to the perspective image the equirectangular image is continuous in all direction, in ~\cref{sec:approach:definition} we analyze and define the panoramic optical flow.
%
To solve the equrictangular image is distortion distortion especially at the top and bottom, this method hires gnomonic projection~\cite{?} to project the equirectangular image to tangent image~\cite{?}, and base on cubemap and regular icosahedron vertex distribution to uniform sample the equirectangular image ~\cref{sec:approach:projstit} to solve the problem~\cref{sec:approach:projstit}.
%
At the same time, to compensate the tangent images narrow field of view, the global rotation operation is used on target image $I_{t+1}$ align to $I_t$ ~\cref{sec:approach:warping}

\subsection{Panoramic Optical Flow Definition}\label{sec:approach:definition}

The panoramic image is enclosed in any direction on the image and the coordinate is continuous, e.g. the pixel location is overflowed the image width will wrap around to another side of the image.
The panoramic optical flow succeeds the features, the pixel's motion vector can overflow the image size.
%It's different from perspective image optical which ~\cite{??}. 
This is one reason the perspective image optical flow method can not track well the pixel moving out the equirectangular image boundary, because it does not suppose the coordinate is wrap-around, ~\cref{?}
\TODO{Add image two show the warp-around optical flow and un-warp-around optical flow.}
%
But it will introduce ambiguity to the panoramic optical flow, 
%the stright line in the 3D space will project to a segement on one great circle.
there are two paths from the source point to the target point along the great-circle on the sphere ~\cite{??}.
To define the panoramic optical flow on equirectangular image format, we propose the following assumes:
the pixel motion is always moving along the the shortest path of great circle;

% in the ERP image range ?
The panoramic image projection to equirectangular image, the pixel along the x-axis is not continuous, but along the y-axis is still continuous. When the pixel position si beyond the  ${Imagewidth -1}$, the pixel position will convert to left again.
But the pixel along the y-axis is continuous.
the pixel motion is always along , which is not larger than 0.5$\pi$.

\TODO{Summarize the benefit of this expression? Why use the warp-around optical flow?}
Sample is continue 
visuzalizion friendly
close the spherical coordinate expression

\begin{figure}[hbt!]
	\centering
	\includegraphics[width=0.45\linewidth]{images/wrap-around-0.jpg}
	\caption{Panoramic Optical Flow.}
	\label{fig:app:warparound}
\end{figure}


\subsection{Optical Flow Warping}\label{sec:approach:warping}


The $f(u,v)$ compose with $f_1(u,v)$ and $f_2(u,v)$, which is the displacement in image's horizontal direction $x$ and vertical direction $y$ respectively.
Optical flow 
$I_{tar}(u',v')$ is corresponding the pixel $I_{src}(u + f_1(u,v), v + f_2(u,v))$.
The optical flow $\textbf{f}$ is estimated pixel displacement vector $f(u,v)$ for each pixel $(u,v)$ in a sequence image pair from source image $I_{src}$ to target image $I_{tar}$.

The tangent image which projected from a distorted image to a plain image have been proposed image segmentation and on ~\cite{eldercvpr2020}

But comparing to a panoramic image the tangent image's available viewport is small, not larger than $\pi$ and just $\frac{\pi}{2}$ in CubeMap projection.

Inspired by the warp base optical flow \cite{?}, and coarst to fine 

Use panoramic images optical flow estimation rotation to roughly align the two images.

\todolist{How to convert the rough optical flow to rotation?}

The image warp, the rotation to warp the image, alignt the 

The optical flow warping, to warp the final optical flow to restore the original images optical flow.


\subsection{Projection \& Stitching}\label{sec:approach:projstit}


The $\hat{\mathcal{F}_t}$ and $\tilde{\mathcal{F}}_t$ is panoramic optical flow stitching from CubeMap and Icosehedron face image (tangent image) optical flow.

The methed composes with 3 steps:
 equirectangular image projection project part of the images pair to perspective images, e.g. tangent images, respectively to get ride of distortation;
 perspective image optical flow estimation to calcuate each tangent image optical flow;
 and optical flow sitching to blend each faces' optical flow back to equirectangular format. 
 
 The perspective image optical flow method is off-the-shelf method which is general to any optical flow method.

The RGB image projection project the point from sphere surface to tangent plan to generate perspective image.~\cite{??}
The optical flow stitching 

\textbf{Equirectangular Image Projection}

% why 

The equirectangular RGB image projection use gnomonic projection to generate the tangent images ~\cite{??}, which is perspective image doesn't have distortation.

When hire the gnomonic porject to project the sphere surfce to know the 

To make the tangent images uniform sample the sphere surface, the tangent points should uniformly distribute on the surface of the unit sphere.~\cite{??}
For cover differen we select the 
The regular polygon ~\cite{??} 

% how to select the samplin points?

Set the tangent point $S$ as the centre point of the projection, who longitude and latitude is $(\theta_0, \phi_0)$,
The sphercial points is $(\theta, \phi)$

project the panoramic image to perspective images.

%Tangent/Center points selection.
Use cubemap or Regular icosahedron, which is a convex polyhedron with 20 faces, 30 edges and 12 vertices.

For the perspective image to cover different field of view (FoV)  to estimate the different scale motion, the sampling point selection regular cubemap and regular icosahedron's inscribe sphere points, which are the most common regular polygon and have different FoV. ~\cite{xxx}

The cubemap's face has wider FoV than the icosahedron's face, which can estimate a larger span motion vector from the tangent image, but Its PPI is smaller than the icosahedron's when they are in the same resolution.

\TODO{CubeMap and ICO's face FOV range. And diagram to show the .}

\textbf{Perspective optical flow stitching}

Each tangent image optical flow estimation is separated, make the faces overlap area's optical flow is not consistent.
So when stitch each faces optical flow to equirectangular format, we compute a confidence weight to each perspective images. 

\begin{equation}
W_{face} =
\end{equation}


The weight compose with 

\TODO{How to compute the weight and blending? Euqation, and Visualized different weight and blend result, and show each face's optical flow.}

