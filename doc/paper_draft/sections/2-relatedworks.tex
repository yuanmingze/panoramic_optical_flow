
\section{Related Works}


\subsection{Equirectangular Image}


\subsection{Cube Map}


\subsection{Geodesic Grids}



\subsection{Tangent Image}


Spherical Coordinate system.



\subsubsection{Gnomonic Projection}

\textbf{Sphere 2 Perspective Image}

%Set the point $S$ as the centre point of the projection, who longitude and latitude is $(\theta, \phi)$

For a projection with the central point $(\theta_0, \phi_0)$,



\begin{itemize}
	\item Ohttps://mathworld.wolfram.com/GnomonicProjection.html
	\item Another entry in the list
\end{itemize}



\subsection{Panoramic Image Processing}

Because of non-linear project between the scene and image, there is distortion in the image, especially in the poles of panoramic image.
How to git ride of the distortion is key point of panoramic image processing method. 

\textbf{Icosahedron}

Icosahedron is used to averagely sampling from the panoramic image, to make less overlap area between each sampling area.

\citet{eder2019mapped}




%\subsection{Undistortion}

\textbf{Gnomonic Projection} ~\citet{gnomonicprojection}

The methods use gnomonic projection project part of sphere to tangent plane, which preserve geodesics and have little distortion near the central point.
So the tangent plane is hired to mitigate the influence of the panoramic image  distortion on the perspective image processing methods.

\citet{coors2018spherenet} use Gnomonic Projection to directly sample data from panoramic image, in this way they transform the omnidirectional sampling to perspective's.

\textbf{Cube Map}


\subsection{Panoramic Dataset}


SUMO

Matterport3D

