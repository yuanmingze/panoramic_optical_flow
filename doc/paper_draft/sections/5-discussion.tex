\section{Conclusion}

We proposed a flexible method for estimating 360° optical flow using tangent images and global rotation warping.
%
Tangent images use gnomonic projection to uniformly sample a 360° ERP image from an initial cubemap to the final icosahedron-based tangent images to overcome the equirectangular projection distortions.
Any existing off-the-shelf optical flow method can be used on the tangent images.
%
Our method uses global rotation warping to pre-align the input images and overcome large motions between frames.
Finally, we blend optical flow to reduce discontinuities at face boundaries.
%
%The result shows our method dramatic improve optical flow in 360° images and achieve state-of-the-art performance.
Our evaluation shows that our method achieves state-of-the-art performance for 360° optical flow estimation.


\new{Our proposed approach overcomes the limited resolutions supported by CNN-based methods due to model size constraints (e.g. 1024$\times$512 \cite{ArtizZAD2020}) by processing tangent images separately.
This enables the estimation of high-resolution 360° optical flow, for example at 2K$\times$1K resolution, which is not supported by existing methods.}


%%\textbf{Limitation \& Future Work}
\new{\paragraph{Limitations.}
Large pose changes are a common limitation of most optical flow methods, which are primarily designed for handling small motions.
In our method, we use global rotation alignment to effectively minimise the net disparity across the unit sphere.
This helps in the case of moderate pose changes, but large pose changes remain challenging.
%\TODO{Maybe? My current optical flow stitch code can not support large motion, because the Gnomonic Projection. If the pixel move to the back of the OpenGL camera, the gnomonic projection based optical flow code will generate wrong optical flow.}
}



\paragraph{Future Work.}
%To sccelerate the projection and stitch step, with multi-thread or c++.
%Pixel-based warping to deal with non-rigid or un-egomoiton.
%
%% CR: not an interesting limitation ->
%To reduce the runtime we will use the c++ and multi-thread to replace the current python and single thread implementation.
%To improve our method performance on camera translation motion, we will replace the global warping method with the pixel-based warping method which will introduce both rotation and translation to global warping.
Our method struggles with large camera translations; one could replace the global rotation warping with a meshgrid-based warping method that can pre-align both rotational and translational camera motions.


% TODO: acknowledgements
\paragraph{Acknowledgements.}

We thank the reviewers for their valuable feedback that has helped us improve our paper.
This work was supported by an EPSRC-UKRI Innovation Fellowship (EP/S001050/1) and RCUK grant CAMERA (EP/M023281/1, EP/T022523/1).
