\section{Conclusion}

We proposed a flexible method for estimating 360° optical flow using tangent images and global rotation warping.
%
Tangent images use gnomonic projection to uniformly sample a 360° ERP image from an initial cubemap to the final icosahedron-based tangent images to overcome the equirectangular projection distortions.
Any existing off-the-shelf optical flow method can be used on the tangent images.
%
Our method uses global rotation warping to pre-align the input images and overcome large motions between frames.
Finally, we blend optical flow to reduce discontinuities at face boundaries.
%
%The result shows our method dramatic improve optical flow in 360° images and achieve state-of-the-art performance.
Our evaluation shows that our method achieves state-of-the-art performance for 360° optical flow estimation.

%%\textbf{Limitation \& Future Work}
\textbf{Future Work.}
%To sccelerate the projection and stitch step, with multi-thread or c++.
%Pixel-based warping to deal with non-rigid or un-egomoiton.
%
%% CR: not an interesting limitation ->
%To reduce the runtime we will use the c++ and multi-thread to replace the current python and single thread implementation.
%To improve our method performance on camera translation motion, we will replace the global warping method with the pixel-based warping method which will introduce both rotation and translation to global warping.
Our method struggles with large camera translations; one could replace the global rotation warping with a \CR{mesh? meshgrid?} pixel-based warping method that can pre-align both rotational and translational camera motions.


% TODO: acknowledgements
