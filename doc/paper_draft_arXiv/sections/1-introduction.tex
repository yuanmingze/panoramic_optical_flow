\section{Introduction}
\label{sec:intro}

Commercial 360° video cameras have recently grown in popularity.
Companies such as GoPro, Insta360, Ricoh, and many others, are now offering affordable consumer 360° video cameras.
At the same time, support for 360° videos has been added across the content creation and consumption pipeline, including video editing software, such as DaVinci Resolve and Adobe Premiere Pro,
video distribution channels such as YouTube, Facebook and Vimeo,
video players like VLC, and virtually all current VR headsets.


Optical flow is a crucial component for processing and editing 360° images and videos as it provides correspondences over time.
Among other things, this enables 360° video stabilisation \cite{Kopf2016}, depth estimation \cite{ZioulKZAD2019}, and novel-view synthesis \cite{BerteYLR2020}.
However, the most common format of 360° images, equirectangular projection (ERP), suffers from severe distortions when mapping a spherical 360° image to the 2D image plane.
This distortion is not considered by most existing optical flow algorithms,
which may lead to flow estimation errors.


Previous work has adapted existing CNN-based techniques designed for perspective images to ERP images by transforming convolution kernels to match the projection \cite{CoorsCG2018, SuG2019, TatenNT2018}.
To overcome the limited spatial resolution supported by these methods, other work proposed extracting, processing and recombining tangent images \cite{EderSLF2020, LuoZSX2019, ZhangLSC2019, LeeJYJY2019, WangHCLYSCS2018, WangYSCT2020}, which supports both CNN-based and traditional methods.
Moreover, these tangent images are based on \emph{gnomonic} projection, which is a common tool used in geography to map a sphere to a plane, as the great circle on the sphere maps to a straight line in the tangent image~\cite{EderSLF2020}.
This principle has been applied to tasks such as monocular depth estimation \cite{WangHCLYSCS2018, WangYSCT2020}, semantic segmentation \cite{EderSLF2020, ZhangLSC2019}, and classification \cite{EderSLF2020, LuoZSX2019, LeeJYJY2019}, but not yet to optical flow.
Our approach builds on this insight to estimate accurate 360° optical flow for high-resolution ERP images by sampling tangent images uniformly based on regular polyhedra (cube and icosahedron).
\looseness-1


This paper makes the following technical contributions:
\begin{enumerate}[nosep]
\item 
We propose a 360° optical flow method based on gnomonic projection to overcome the distortions induced by the equirectangular projection of 360° images.

\item
We integrate a global warping method to align large rotations between 360° images.

\item
We introduce a tangent image optical flow blending method that can dramatically reduce discontinuities at face boundaries.

\item 
We refine our flow estimates by increasing the sampling density of tangent images from an initial cubemap to the final icosahedron-based uniform sampling of tangent images.
\end{enumerate}
